\title{Statistics}

{{navbar}}

\subsubsection{Statistics}

The \texttt{edward.stats} library provides a collection of primitive
distribution methods for use in TensorFlow.

{{autogenerated}}

\begin{lstlisting}[language=Python]
class Distribution(object):
  """Template for all distributions."""
  def rvs(self, size=1):
    """
    Parameters
    ----------
    size : int, list of int, or tuple of int, optional
        Number of samples, in a particular shape if specified in a
        list or tuple with more than one element.

    params : float or np.ndarray

    Returns
    -------
    np.ndarray
        np.ndarray of dimension (size x shape), where shape is the
        shape of its parameter argument. For multivariate
        distributions, shape may correspond to only one of the
        parameter arguments, e.g., alpha in Dirichlet, p in
        Multinomial, mean in Multivariate_Normal.

    Notes
    -----
    This is written in NumPy/SciPy, as TensorFlow does not support
    many distributions for random number generation. It follows
    SciPy's naming and argument conventions. It does not support
    taking in tf.Tensors as input.

    The equivalent method in SciPy is not guaranteed to be
    supported with a batch of parameter inputs, e.g., a vector of
    location parameters in a normal distribution, or a matrix of
    concentration parameters in a Dirichlet. This is.

    This does not follow SciPy's behavior, e.g., the number (or
    shape) of the draws will always be denoted by its outer
    dimension(s).

    params as a 2-D or higher tensor is not guaranteed to be
    supported (for either univariate or multivariate
    distribution).

    size as a list or tuple of more than one element is not
    guaranteed to be supported.

    For most distributions, the parameters must be of the same
    shape and type, e.g., n and p in Binomial must be np.arrays()
    of same shape or both floats. For some, they may differ by one
    dimension, e.g., n and p in Multinomial can be float and
    np.array(), or both np.arrays, and n always has one less
    dimension.
    """
    raise NotImplementedError()
\end{lstlisting}
